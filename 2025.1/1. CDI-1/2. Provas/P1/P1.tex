\dfrac{num}{den}\documentclass{exam}
\usepackage{amsmath}

\title{Prova Avaliativa 1 - Fabio\footnote{Créditos: Leandro Colaes, Henrique B., Marcus Vinicius, Bruno Kenzo.}}
\date{}
\begin{document}
\maketitle

	\textbf{1. Resolva os limites: }
	\begin{align*}
		\textbf{a)}\lim_{x \to -\infty} \frac{x -1}{\sqrt{x^{2}-1}} &, &
		\textbf{b)}\lim_{x \to 2} 1 + \frac{\sqrt{x}-\sqrt{2}}{x-2} &, &
		\textbf{c)}\lim_{x \to 0} \frac{\tan{x}\cdot{\sec{2x}}}{3x} &, &
		\textbf{d)}\lim_{x \to 2} \frac{x^{2}-x-2}{x^{3}-8}
	\end{align*}\\

	\textbf{2. Explique se as afirmaçãões são verdadeiras ou falsas:} 
	
	
	~~~~\textbf{a) Se um $f^{'}$ existe, então existe a função $f$ é continua em x.}

	~~~~\textbf{b)Se uma $f$ é impar ($-f(x) = f(-x)$), então a $f^{'}$ é par ($f(-x) = f(x)$)}\\
	
	\textbf{3) Use a teorema do confronto para resolver as questões seguintes:}
	\begin{align*}
		\textbf{a)}\lim_{x \to 0} x^{2}\sin{\frac{1}{x^{2}}} &, &
		\textbf{b)}\lim_{x \to c} 	{|f(x)| = 0} \implies \lim_{x \to c}	f(x) = 0	
	\end{align*}\\
	
	\textbf{4) Deriva as sequintes funções:}
	\begin{align*}
		\textbf{a)}\sin{x}\cdot\cos{x} &, & 
		\textbf{b)} &, &  
		\textbf{c)} (x^{3}-2^{2}+1)^{3} &, &
		\textbf{d)} \left(\frac{x}{x+1}\right)^{10}
	\end{align*}\\
	
	\textbf{5) Encontre os pontos onde exista a tangente horizontal da função $\cos{2x} + 2\cos{x}$.}
	
	
	
\end{document}