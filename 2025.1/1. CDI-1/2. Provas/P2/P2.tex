\documentclass{exam}
\usepackage{amsmath}

\title{Prova Avaliativa 2 - Fabio\footnote{Créditos: Leandro Colaes}}
\date{}
\begin{document}
\maketitle

\textbf{1.} $y = 3x²-5$, $\Delta$ seja um incremento de $x$, mostre a formula geral de $\Delta y$ e $dy$ \\

\textbf{2.} Faça as implicitas 
\begin{align*}
	\textbf{a)}x^{2}y^{3}+xy^{2}+5x-5 = 0 &, &
	\textbf{b)} x^{2}\sin{x}
\end{align*}

\textbf{3.} $y = 2\sin{x}-\cos2x$. Identifique os extremos locais no domínio $[0, 2\pi]$. \\

\textbf{4.} Dada a função $y = x^{5}-5x^{3}$, identifique os pontos críticos, os extremos locais e a concavidade. \\

\textbf{5.} Pela função $f(x) =$ \fontsize{14}{\baselineskip}$\frac{2x}{9-x^{2}}$, analise o domínio, a continuidade, simetria, pontos de inflexão e extremos e a assíntona.

\end{document}