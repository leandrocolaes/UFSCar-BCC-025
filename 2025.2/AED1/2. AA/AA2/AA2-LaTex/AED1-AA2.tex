\documentclass{exam}
\usepackage{amsmath, amssymb, tikz, hyperref, amsthm}
\usepackage[portuguese]{babel}
\title{Introdução à grafos}
\date{}
\begin{document}
\textbf{1)b}

A expressão para obter a média dos casos, e por consequência, o caso médio é $$T(n) = \frac{1}{n-1}\sum_{k=1}^{n-1}\sum_{i=1}^{k}i.$$ A intenção é percorrer todas as possibilidades, desde o melhor caso que é $n-1$ até o pior caso que é $(n-1) + (n-2) + (n-3) + ... + 1$.\\

Separaremos a equação em $3$ partes para uma melhor compreensão:

A: $\frac{1}{n-1}$, como a equação é uma média, será necessário dividir pelo o número total de eventos.

B: $\sum_{k=1}^{n-1}$, é a quantidade de passadas que o algoritmo pode dar.

C: $\sum_{i=1}^{k}i$ será usado para auxiliar na diminuição de comparação, pois a cada passada o algoritmo ele posicionará corretamente o elemento de maior valor.

Simplificando a equação: 
\begin{align*}
	T(n) &= \frac{1}{n-1}\sum_{k=1}^{n-1}\frac{k(k+1)}{2}\\
	 &= \frac{1}{2(n-1)}\left[
	\sum_{k=1}^{n-1} k^{2}+\sum_{k=1}^{n-1}k
	\right]\\
	&=\frac{1}{2(n-1)}\left[
	\frac{n^3-1}{2} - \frac{n(n-1)}{2} - \frac{n-1}{3} + \frac{n(n-1)}{2}
	\right]\\
	&=\frac{1}{2(n-1)}\left[
	\frac{n^3-1}{3} - \frac{n-1}{3}
	\right]\\
	&=\frac{n^2+n}{6}\\
\end{align*}

Ou seja, o $T(n)$ é equivalente a $O(n^2)$

\textbf{2)b}
\begin{align*}
	T(n) &= \frac{1}{n}\sum_{k=1}^{n}\left[
		\sum_{i=1}^{k} \left\{
			2 + \sum_{j = i + 1}^{n}2
		\right\}
	\right]\\
	 &= \frac{1}{n}\sum_{k=1}^{n}\left[
		2k + 2nk-k^2-k
	\right]\\
	&= \frac{1}{n}\left[
		\sum_{k=1}^{n}k + 2n\sum_{k=1}^{n}k-\sum_{k=1}^{n}k^2
	\right]\\
	&= \frac{1}{n}\left[
		\frac{n(n+1)}{2} + \frac{2n^2(n+1)}{2} - \frac{n(n+1)(2n+1)}{6}
	\right]\\ 
	&= \frac{2n^2}{2} + n + \frac{1}{3}
\end{align*}

\textbf{3)b}
\begin{align*}
	T(n) &= \frac{1}{n-1}\sum_{k=1}^{n-1}\left[
		\sum_{i=2}^{k+1} \left(
			1 + \sum_{j=1}^{i-1}
		\right)
	\right]\\ 
	&= \frac{1}{n-1}\sum_{k=1}^{n-1}\left[
		k^2+2k
	\right]\\
	&= \frac{1}{n-1}\left[
		\sum_{k=1}^{n-1}k^2 + 2\sum_{k=1}^{n-1}1
	\right]\\
	&=\frac{1}{n - 1}\left[
		\frac{(n-1)(n^2+n+1)}{3}-\frac{n(n-1)}{2} -\frac{n-1}{3} + n(n-1)
	\right]\\
	&=\frac{1}{3}n^2 + \frac{5}{6}n
\end{align*}

Assim, é equivalente ao $O(n^2)$
\end{document}